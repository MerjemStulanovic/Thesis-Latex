\section{Technical Features of the Backend Implementation}\label{sec:specifics-backend-implementation}
In this section we will concentrate on the features of the backend implementation of the project. The structure of the backend will be mentioned,
along with the detailed explanation of the main functionalities implemented in the application.

\subsection{Structure}\label{subsec:structure}
Backend of the application is structured as a 3-layer architecture composed of controllers, services, and repositories. In addition, DTO (data
transfer object) classes are defined to facilitate communication with the frontend.
\clearpage

Next we will provide a list of controllers used in the implementation and briefly describe their usage:
\begin{itemize}
    \item Artist controller -- provides the options to get all artists, create a new artist, edit and delete the existing artist.
    \item City controller -- provides the options to get all cities, create a new city, edit existing city's name, delete the city. In addition, this controller also handles cropping the map. One route returns the coordinates of a cropped map within the full map, and
            a second one allows the frontend to retrieve the image of the cropped map.
    \item Exhibition controller -- provides the options to get all exhibitions, edit or delete existing exhibitions.
    \item Historic event controller -- provides identical options as the exhibition controller.
\end{itemize}

The service layer in most cases just routes the methods to the repositories and is not described in detail. One exception is the CityMap service
which handles the cropping of the map explained in \Cref{subsec:map-cropping}.

Overall, this is the list of the entities found in data layer:
\begin{itemize}
    \item Artist
    \item City
    \item Residence -- contains information about cities artists lived in and the corresponding time period
    \item Exhibition
    \item Historic event
\end{itemize}

A detailed definition of these entities can be found in \Cref{ch:database-structure}.

\subsection{Map Cropping}\label{subsec:map-cropping}
As mentioned before, the whole map is not shown in the visualization of the cube. The map is cropped to the portion which includes all the cities
that need to be presented for the visualization of the artists' life paths. Cropping the map is a somewhat specific operation in the application, so
it will be explained in detail.

Frontend communicates to the backend which cities are required. Initially, the cities' positions on the map are saved as percentages relative to the
size of the complete map on the screen. The first step is to convert these units from the percentages to pixels. Then, we determine the smallest x
and y coordinates, as well as the largest. This represents the beginning and the ending corners of the cropped map. Based on this, the width and the
height of the map are easily determined. However, since these values are calculated according to the locations of the cities, it is not guaranteed
that the width and the height of the cropped region will be the same. To make sure that we get a square required for the cube, additional padding
might be necessary, as illustrated in \Cref{fig:figuremapcropping}.

\Cref{fig:figuremapcropping} presents us with a visual solution to this problem. We start by determining the bigger of the two dimensions, which is
in our case the width. For the smaller of those two, it needs to be increased to the size of the larger one to get a square. This is done by adding
padding on both sides of the smaller one. In order to add the padding, the smaller dimension is subtracted from the bigger and the result is divided
by two. This is the padding that needs to be added to the start coordinates -- City 1 in the figure (in this case, the padding is subtracted from the
start coordinates because we need the new coordinates which are smaller than our start ones). The result coordinates are marked as S'. The same
is done for the end coordinates -- City 2 in the figure, we add the padding because we need the new coordinates which are further from our end ones.
The result coordinates are marked as E'. From now on, we assume that the start and end coordinates for both axes are brought to the same form -- the
width and the height are now the same.

Additionally, it is not visually pleasing, as seen in \Cref{fig:figuremapcropping} to have the start and end cities' coordinates on the edges of the
cropped map, so a small padding is added here as well. The procedure of adding this padding is similar to the one already explained (subtracting
the padding from the start coordinates, and adding the padding to the end coordinates). Then, based on the values of the new start and end
coordinates in pixels, the map is cropped accordingly and sent to the frontend. However, as already mentioned, the frontend also needs to know the
position of the start and end coordinates of the cropped map relative to their position on the original map. Because of this, the start and end
coordinates of the cropped map are divided by the width and the height of the original map in order to get the corresponding values that represent
the positions of the coordinates on the original map in percentages. These values are also sent to frontend as bounds and based on them, the frontend
can calculate where the correct positions of the cities on the cropped map is. This was explained in \Cref{subsec:technical-specifics-stc} and seen
in \Cref{fig:figure3.13}. Code example for this functionality can be found in \Cref{ch:code-examples}.

\begin{figure}[hbt!]
    \begin{center}
        \begin{tikzpicture}
            \draw[-] (-5, -5) -- (-5, 5);
            \draw[-] (-5, 5) -- (5, 5);
            \draw[-] (5, -5) -- (5, 5);
            \draw[-] (-5, -5) -- (5, -5);
            %
            \draw[dashed] (-3, -3) -- (-3, 3);
            \draw[dashed] (-3, 3) -- (3, 3);
            \draw[dashed] (3, -3) -- (3, 3);
            \draw[dashed] (-3, -3) -- (3, -3);
            %
            \draw[-] (-3, -1.8) -- (-3, 1.8);
            \draw[-] (-3, 1.8) -- (3, 1.8);
            \draw[-] (3, -1.8) -- (3, 1.8);
            \draw[-] (-3, -1.8) -- (3, -1.8);
            %
            %\node[] at (-1, -0.2) {\footnotesize $(0.4,0.5) \mapsto (0.\overline 3,0.5)$};
            %
            \node at (-5,5) [circle,fill,inner sep=1.5pt]{};
            \node[] at (-5.4, 5) {S};

            \node at (5,-5) [circle,fill,inner sep=1.5pt]{};
            \node[] at (5.4, -5) {E};

            \node at (-3,3) [circle,fill,inner sep=1.5pt]{};
            \node[] at (-3.4, 3) {S'};

            \node at (3,-3) [circle,fill,inner sep=1.5pt]{};
            \node[] at (3.4, -3) {E'};

            %
            \node at (-3,1.8) [circle,fill,inner sep=1.5pt]{};
            \node[] at (-3.7, 1.8) {\footnotesize City 1};

            \node at (1.5,0.6) [circle,fill,inner sep=1.5pt]{};
            \node[] at (1.5, 0.3) {\footnotesize City 3};

            \node at (-0.6,-0.5) [circle,fill,inner sep=1.5pt]{};
            \node[] at (-0.6, -0.9) {\footnotesize City 4};

            \node at (3,-1.8) [circle,fill,inner sep=1.5pt]{};
            \node[] at (3.7, -1.8) {\footnotesize City 2};
        \end{tikzpicture}
    \end{center}
    \caption{Map cropping and size adjustment}
    \label{fig:figuremapcropping}
\end{figure}

\clearpage
