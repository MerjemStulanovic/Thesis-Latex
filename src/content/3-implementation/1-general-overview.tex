\section{General Overview}\label{sec:general-overview}
Before every part of our implementation is explained in detail, some general information about the visualization will be mentioned. As it will be detailled in \Cref{ch:background},
a web application for visualizing the space-time cube was implemented. It consists of frontend and backend parts. For the frontend part, Angular,
Three.js, and WebXR technologies were used, and for the backend part, Spring Framework.

\subsection{Backend}\label{subsec:backend}
The task of the backend was to manage the data -- save the information the user enters and provide the appropriate ones during the creation of the
cube. Since the cube is a spatio-temporal representation, for the spatial part some sort of a map was needed. The users enter the required cities on the
entire map by themselves, and the backend part manages the selection of the right portion of the map based on the provided cities for the chosen artists.
Including the whole map of the world in the cube would not be practical as only the part the artists lived in is important. More about this
concept will be discussed later in \Cref{subsec:map-cropping}.

The data types included in our database, not counting the helper tables, are \texttt{Artist}, \texttt{City}, \texttt{Residence}, \texttt{Exhibition}, and
\texttt{HistoricEvent}. \texttt{Artist} contains the name, the surname, and the list of locations artists lived in during their life. \texttt{City}
contains the city name and the city's x and y coordinates on the map. \texttt{Residence} contains the city name, the start and end date, the number of
artworks the artist created during their stay, and a label to describe if the stay was important and why. \texttt{Exhibition} contains the title of the
exhibition, the city where it was held, its start and end date, and the list of artists who attended it. \texttt{HistoricEvent} entity looks very similar,
but instead of the list of the artists, it contains the list of the cities where the event took place. More about the structure of the backend is
available in \Cref{sec:specifics-backend-implementation}.

\clearpage

\subsection{Frontend}\label{subsec:frontend}
The frontend is the web application which the user sees and interacts with. The application gives access to several functionalities such as:

\begin{itemize}
    \item managing the cities on the map,
    \item managing the list of artists,
    \item managing the exhibitions artists attended,
    \item managing the historical events that may have influenced artists' work.
\end{itemize}

For the presentation of the cube, the user has two options:

\begin{itemize}
    \item running the representation in a desktop web browser,
    \item running the representation in an AR viewer for mobile devices.
\end{itemize}

Everything on the frontend part is managed by the user. None of the data is taken via external sources, so the user has full control of it.
