In order to find the right way of representing an artist’s life, we have reviewed previous works and ways of doing it.
This chapter will explore the information visualization techniques used in the past that could help in finding the
perfect one for our case, trying to summarize as many crucial techniques as possible. We will try to highlight their
strengths and weaknesses and improve the ones that fit the best for our scenario. The goal is to find the information
visualization techniques which represent the artist’s life and its most important moments in the best possible way.

Various techniques help us do it, including a whole range of charts, graphs, tables, maps, and many more. We can see a specter of
colors and patterns which capture the very essence of what collected data can show us. Although there are currently many
information visualization techniques, we will mostly focus on spatio-temporal representations that capture people's actions through space
and time. This category of data representation techniques is particularly interesting for this research as its goal is
to present the life of an artist, including places they lived, the timespan of their life, and some interesting
connections to other artists. We will explore as many aspects of their lives as possible and focus on the most
important ones.

First, a few paragraphs will be dedicated to the history of information visualization and some of the most important
examples which appeared through time.
