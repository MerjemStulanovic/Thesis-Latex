\section{Strengths and Weaknesses of the Space-Time Cube}\label{sec:strengths-weaknesses-stc}
Before proceeding to the next chapter of this research, we need to address points of strength and weakness of the space-time cube as this
visualization technique will be used in the implementation part. We have seen multiple examples of space-time cube and their usage in
different domains. However, this visualization has its strengths and weaknesses like any other. We briefly mentioned some of them in
\Cref{sec:space-time-cube}, but we will present them in detail here and try to understand where improvements can be made.
Since we want to display various aspects related to an artist’s life, the objective of the visualization is to implement multidimensional data
while keeping the user’s cognitive load low.

Strengths of the space-time cube:
\begin{itemize}
    \item Proved to have unique strengths in showing the multidimensional data~\citep{windhager2016reframing}.
    This property is of great importance and the space-time cube permits to implement multiple dimensions without
    losing its efficiency.
    \item Integration of data into the spatio-temporal dimension is fair and balanced because the most effective variables are distributed
    equally to all sides~\citep{windhager2018orchestrating}.
    \item Can mediate between different temporal visualizations and translate from temporal to spatial perspectives
    while allowing seamless transitions. This makes it act as a translational hub or operational cognitive
    scaffold~\citep{windhager2018orchestrating}.
    \item Spatio-temporal patterns and clusters are identified more quickly and accurately with the space-time cube than with
    the 2D visualization~\citep{amini2014impact, kristensson2008evaluation}.
    \item Visual variables for 2D maps (size, shape, color, value, orientation, and texture) by Bertin~\citep{bertin1983semiology}
    can be applied to the space-time paths to represent quantity or quality differences of attribute information
    (color for distinguishing multiple space-time paths, path width for quantitative characteristics, and additional attachments to the
    space-time paths whose size, color, and shape represent quantity or quality difference of attributes)~\citep{song2021visualization}.
\end{itemize}

Weaknesses of the space-time cube:
\begin{itemize}
    \item Investigating and visualizing large collections in the space-time cube significantly increases the response time
    of the analysts~\citep{windhager2020many}.
    \item Visual clutters are formed when the number of space-time paths is larger than the capacity of the cube,
    therefore resulting in a small amount of efficient information that can be analyzed~\citep{song2021visualization}.
\end{itemize}

With this we have concluded the research on previous works. In the following chapter, the conceptual part of the thesis proposal which improves the
currently available implementations will be presented.
