\section{Geospatial Data Visualization}\label{sec:geospatial-data-visualization}

If we talk about space and the need to represent data in it, we are most likely talking about maps and cartography.
Nowadays, we are surrounded by them and they help us vastly in many areas of life. With the evolution of geographic
information system (GIS) in the 1960s, the notion of maps got even broader applications. GIS is widely used by
different institutions, from academic to government agencies and corporations~\citep{longley2005geographic}.
We will later consider those representations that help us understand people's movements on Earth. Exploring these data visualizations will
help us have a better approach to the research of this work.

Since our goal is to find a way to represent an artist’s life both in terms of places where they lived and the
timespan of their life, we can divert our attention to the spatio-temporal data visualization techniques, combining the concept
of space and time.

