\section{Methodology and Solution Approach}\label{sec:methodology-solution-approach}
This section will cover the methodology and solution approach of the thesis research. We can divide our work into two parts, a literature
review of the previous works related to the thesis, and an implementation part based on the gathered knowledge.

In the literature review, we start from the early examples of information visualization and gradually present several categories like temporal,
geospatial, and spatio-temporal visualizations before we explore the one which will be used in our implementation, the space-time cube. We
address its points of strength and weakness, evaluations conducted on it, and see its applications in the cultural heritage domain. Another
important part of the literature review is related to the research on an artist’s life and the ways it is presented. This part helps us in
understanding which aspects of artists' lives are important to visualize and which visualizations help us do that in the best possible way.

For the implementation, several technologies were chosen to visualize the space-time cube. Angular, Three.js, and WebXR were used for the
frontend part, and the Spring Framework for the backend part. A web application is created that generates the space-time cube from the given data
inputs. An authoring interface is added that enables users/experts to enter their data into the application. In this way, the user is in control
of the system and can insert, edit or delete the data as they see fit, rather than relying on data provided by third parties. The interface can
manage artists and properties related to artists' lives, can filter information to present in the cube, and much more. The users can explore the
visualization in their browser or visualize the augmented reality version of the cube. This approach enhances 3D visualizations based on web
browsers, providing an additional engaging opportunities for exploring an artist's life.

For the evaluation of our work and results, we designed a pilot study proposal for around ten participants. It is composed of several
categories of questions related to the engagement (user's ability to effectively operate the visualization), learning (what knowledge was gained
through the interaction with the visualization), and general questions and comments (these include users' evaluation of the features and
feedback on which ones were well-received and where improvements could be made).
