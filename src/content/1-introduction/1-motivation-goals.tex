\section{Motivation and Goals}\label{sec:motivation-goals}
Visualization of cultural heritage has become recently popular thanks to the development of technology. We used to experience art collections
in person and in places where they are stored, such as museums, galleries, libraries, and archives~\citep{windhager2018visualization}. All
these artworks are visited and observed by people all around the world, with the spotlight being more oriented to the artworks rather than the artists
themselves. In order to understand the background of artists’ artworks and get an insight into their thinking, we need to explore their lives
too. For this reason, in addition to the visualization of artwork collections, it is useful to devote our attention and research to the
visualization of an artist’s life.

Information about the life of an artist is commonly presented in written form. We have biography books and memoirs~\citep{herrera1983frida,
    isaacson2017leonardo, tomkins1999duchamp}, articles, interviews, and others. The problem arises when we want to visualize their life
because there are many aspects and dimensions which can be taken into consideration and presented visually. Too much information may take over
the main goal of any information visualization -- transforming data into something more meaningful and easier to understand.

This research aims to visualize an artist’s life and provide a better understanding of the chosen dimensions, where the main goal is to
visually present their changes through space and time. The aim is to have a better understanding of artists' lives, to focus on undetected facts
and relations which may have influenced their works.

Here, we list the goals we want to accomplish during this research:

\begin{itemize}
    \item Visualize an artist's life path and places they lived in during their lifetime.
    \item Include the possibility to visualize multiple artists' lives and try to capture connections and interaction points, if any.
    \item Provide an interface for entering information about the artist’s life. In this way, the advanced users have complete control over the
            visualization and can create representations that are not available, without relying on third parties.
    \item Explore relevant events the artists might have attended, such as exhibitions that might have influenced their artistic production.
    \item Explore the minor or major historical events that may have influenced their lives and works.
    \item Make the visualization interactive with the possibility for the user/expert to filter the visualization parameters in order to
            focus on specific information.
    \item Combine the visualization with an augmented reality (AR) prototype for an immersive experience.
\end{itemize}

At this point, we would like to briefly explain the novelty of our work. During the research of literature, no examples of interactive space-time cubes that
combine artists' life paths with other aspects such as their work, attended events, etc. were found. Many of the works we surveyed focus on presenting a
single artist, whilst in our visualization, we allow the user to explore multiple lives and examine potential relations among artists. Additionally, none
of the works that we have seen provided manual insertion of data, but our work in contrast provides an easy-to-use interface that allows domain experts to
enter data themselves.

Lastly, we intend to make the first steps in interactive visualization of artists' lives, and present a clean implementation that will serve as
a foundation for further improvements and addition of new features. Based on our results, we hope to encourage research of the spatio-temporal
visualizations, especially in combination with the augmented reality experience.
