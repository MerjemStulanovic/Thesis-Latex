For evaluating the implemented space-time cube visualization, a pilot study test is presented in this chapter and will be conducted in the near
future. A group of around ten people will be considered for participation in the study. They will perform the test
individually. The study is composed of several phases which will be explained in detail in the following sections.

Phases of the pilot study on the  space-time cube visualization:
\begin{enumerate}
    \item Preparation of a short text with information about a life of an artist.
    \item Each participant of the test is given brief information about the goals of the test. Each of them is given a copy of the information to be inserted through the space-time cube authoring interface.
    \item Participants are asked to insert the data from the text they were given.
    \item After inserting the information, participants are asked to test the obtained representation, both the web browser version and the AR version.
    \item In the end, a questionnaire is presented with questions from different facets of the experiment to be investigated such as: engagement questions, learning, feedback questions, and general comments and suggestions about the prototype and the related presentations.
\end{enumerate}

\section{Preparing a Short Text}\label{sec:preparing-short-text}
The first step of the pilot study that will be conducted is to prepare a text which includes information about an artist's life. The same text will
be used for all the participants, in order to have the most consistent results in the end. In our case, a fragment of the life of the painter
Vincent van Gogh will be used. It contains references to several stays during his life, with information about each stay: the date when he arrived
in a certain city as well as the date he left it, whether the stay was important and the description of why, artwork-related information for each
stay, and more. The text also includes an exhibition he attended with other artists.

\section{Inviting Participants to the Test}\label{sec:inviting-participants}
The next step is to invite participants individually to the study. Participants are informed that the test will be conducted within a timeframe of
one hour. The first ten minutes are reserved for briefly explaining the space-time cube visualization, as well as for the participants to read the
text on the artist's life. In the next ten minutes, participants insert the data in the application. After this, twenty minutes are reserved for
exploring the visualization -- the browser version and the AR version. The remaining time is left for filling out the questionnaire.

\section{Inserting the Data}\label{sec:inserting-the-data}
As already mention, the participants are given a short text which contains the information that needs to be inserted in the application. Firstly,
they need to insert cities mentioned in the text in the \emph{Manage cities} page, then insert information about the artist in the \emph{Manage
artists} page. The last step of inserting data is to enter one exhibition which the artist attended in the \emph{Manage exhibitions} page.
Participants do not need to enter data related to historical events, because they will already be available. The reason behind this is to avoid
spending too much time on inserting the data. The goal is to focus more on exploring the visual representation.

\section{Testing the Presentation}\label{sec:testing-visualization}
During the 10-minute testing of the visualization, half of the participants are asked to firstly test the browser version of the cube and then
the AR version. The other half of them firstly tests the AR version and then the browser version. This is done to counterbalance the effect of
exploring one representation before the other.

The participants will explore the life of the artist first, explore the connections to the other
artists (information about the life of other artists will already be available), filter information and more. They will be given the opportunity to
freely navigate the presentation and to perform specific tasks.

\section{Answering the Questionnaire}\label{sec:answering-questionnaire}
The questionnaire the participants will fill out at the end of the study is composed of several categories of questions. They were already briefly
mentioned at the beginning of the chapter, but now we will present all of them in detail followed by the list of the questions for each category.

\subsection{Engagement}\label{subsec:questionnaire-engagement}
The engagement part of the questionnaire gives an insight into the experience the participants had with the application and the visualization. This
segment of the questionnaire was taken from O’Brien et al.~\citep{o2018practical}. It is composed of twelve statements and the participants will
use a 5-point Likert scale shown in \Cref{tab:engagement-scoring-scale} to evaluate each of the statements.

\vspace{0.5cm}

\begin{table}[ht]
    \begin{center}
        \begin{tabular}{ccccc}
            \hline
            Strongly disagree & Disagree & Neither agree nor disagree & Agree & Strongly agree\\ \hline
            1 & 2 & 3 & 4 & 5\\ \hline
        \end{tabular}
    \end{center}
    \caption{5-point Likert Scale}
    \label{tab:engagement-scoring-scale}
\end{table}

\clearpage

Engagement statements:
\begin{itemize}
    \item I lost myself in this experience.
    \item The time I spent using the application just slipped away.
    \item I was absorbed in this experience.
    \item I felt frustrated while using the application.
    \item I found the application confusing to use.
    \item Using the application was tiring.
    \item The application was attractive.
    \item The application was aesthetically appealing.
    \item The application appealed to my senses.
    \item Using the application was worthwhile.
    \item My experience was rewarding.
    \item I felt interested in this experience.
\end{itemize}

\subsection{Learning}\label{subsec:questionnaire-learning}
In this part of the questionnaire, we are interested in what the participants have learned during the interaction with the visualization. The main
goal is to determine what the participants learned from the representation that was not as obvious from just reading the initial text on artist's
life that the participants were given at the beginning of the session.

\clearpage
\subsection{General Comments and Suggestions}\label{subsec:questionnaire-feedback}
In this part of the questionnaire we are interested in what features were appreciated or not, and what could be improved for the future work.

Questions:
\begin{itemize}
    \item Which features were appreciated and why?
    \item Which features were not appreciated and why?
    \item What can be improved or which new features could be added?
    \item Leave further comments about your experience with the application, if any.
\end{itemize}