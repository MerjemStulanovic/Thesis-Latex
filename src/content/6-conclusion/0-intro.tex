Starting from the history of the information visualization techniques, we have explored
different examples of data visualization. Because of the focus of this thesis, the main interest was mostly for spatio-temporal techniques, and in
particular the space-time cube. The space-time cube representation was chosen for visualizing artists' lives as it provided the opportunity to
represent their spatio-temporal trajectory with additional relevant relationships and events in the same visualization. After exploring the
space-time cube representation, we pointed out its strengths and weaknesses before moving on to the implementation of our work.

A web application was created that the user/expert can use for entering information about an artist's life and then run the visualization in a web
browser or in an AR viewer. Various features were added to the cube in order to explore different aspects of the artists' lives. The representation
allows to explore their life path, their artistic works, exhibitions they attended, connections to other artists, as well as historic events that
could have influenced their life activities. Later we discussed the technologies which were used to implement the application. At the end, the
structure of a pilot study to be performed at a later point was created for evaluating the application with detailed structure and a questionnaire
composed of several categories.

\clearpage

While the space-time cube has already been used for other domains, to our knowledge it has never been used for visualizing artists' lives, and
therefore it serves as an interesting tool to be delivered to the hands of art experts or art exhibition holders/participants to augment knowledge
about artists, relationships and events that happened in their lives.

The AR option can be a starting point for future work which is explained in detail in \Cref{sec:future-work}.

The goal of this research was to present the life of an artist in a meaningful and innovative way. Choosing the space-time cube for this was a
challenging research problem since we had to decide which aspects of an artist's life we wanted to include in the visualization. Previous works
mostly included single dimension of a person's life or used multiple cubes to represent separate dimensions. We tried to combine several of the
properties in a single visualization and present the users with the possibility to interact with the visualization as well as filter the information
they were interested in. Additionally, exploring multiple lives in a single cube added new perspectives of identifying connections and relations
between artists.

With this work we hope to contribute to the research of the life of an artist and present new ideas and approaches of how it can be visualized.

