\section{Three.js}\label{sec:three.js}
This section presents the Three.js library. The mentioned information was taken from~\citep{dirksen2013learning}. Also, an
example~\citep{threejsdocs} will be presented of how to create a 3D cube.

Three.js is a JavaScript library and API (application programming interface) for creating 3D graphics in a web browser. It was released by
Ricardo Cabello in 2010 on GitHub. The library provides an easy-to-use JavaScript API around the WebGL features without having to learn WebGL
in depth since programming it directly is very complex and error-prone.

\clearpage

Three.js makes it easy to:
\begin{itemize}
    \item Build both simple and complex 3D geometries.
    \item Animate and move objects in a 3D scene.
    \item Add textures and materials to the objects.
    \item Use various light sources to illuminate the scene.
\end{itemize}

The main components for displaying anything with Three.js are the scene, camera, and renderer, which are used to render the scene with the
camera. Code example below shows how to create a simple 3D cube.

\begin{figure}[hbt!]
    \begin{center}
        \begin{lstlisting}[language=JavaScript,label={lst:threejs-code-1},belowskip=-1 \baselineskip]
const scene = new Scene();
const camera = new PerspectiveCamera( 75, window.innerWidth / window.innerHeight, 0.1, 1000 );
const renderer = new WebGLRenderer();
renderer.setSize( window.innerWidth, window.innerHeight );
        \end{lstlisting}
    \end{center}
    \caption{Main components in Three.js}
    \label{fig:figure4.3}
\end{figure}

The first line creates the scene, which is used to store objects. Then the camera object is created using the \texttt{PerspectiveCamera}.
Three.js has many additional cameras such as \texttt{ArrayCamera}, \texttt{CubeCamera}, \texttt{OrthographicCamera}, but the
\texttt{PerspectiveCamera} is the one that resembles how the human eye sees. There are multiple attributes the \texttt{PerspectiveCamera}
uses: field of view, aspect ratio, near, and far (respectively).

The renderer is defined next, and the library has additional renderers besides the \texttt{WebGLRenderer} for those who use older browsers or do not
have the support for WebGL. After creating the renderer instance, it is important to set the size at which to render the application. This is done by
using the \texttt{setSize()} function.

\clearpage

After the scene is created, it is necessary to create an object to add into the scene.

\begin{figure}[hbt!]
    \begin{center}
        \begin{lstlisting}[language=JavaScript,label={lst:threejs-code-2},belowskip=-1 \baselineskip]
const geometry = new BoxGeometry( 1, 1, 1 );
const material = new MeshBasicMaterial( { color: 0x00ff00 } );
const cube = new Mesh( geometry, material );
scene.add( cube );
camera.position.z = 5;
        \end{lstlisting}
    \end{center}
    \caption{Creating an object in Three.js}
    \label{fig:figure4.4}
\end{figure}

\texttt{BoxGeometry} creates a cube, and in addition to creating the cube, we need a material to color it. \texttt{MeshBasicMaterial} is one
of the available materials in the library, and it contains attributes such as: the color of the material, how reflective it is, whether it has
transparency, whether it emits light, how metallic it is, whether it should use a texture, etc. This allows us to easily define materials which mimic
their real world counterparts.

Lastly, everything is tied together by using a \texttt{Mesh} object to merge the geometry and corresponding material so
that it is possible to insert it into the scene. For this, the created object is added to the scene using \texttt{scene.add()}. To avoid the camera
and the cube being inside each other, the camera has to be moved out in a proper position.

After the scene and the object within it are specified, the last thing to do is to render the scene using the function \texttt{animate()}.

\begin{figure}[hbt!]
    \begin{center}
        \begin{lstlisting}[language=JavaScript,label={lst:threejs-code-3},belowskip=-1 \baselineskip]
function animate() {
    requestAnimationFrame( animate );
    renderer.render( scene, camera );
}
        \end{lstlisting}
    \end{center}
    \caption{Rendering the scene in Three.js}
    \label{fig:figure4.5}
\end{figure}

\clearpage
