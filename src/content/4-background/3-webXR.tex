\section{WebXR}\label{sec:webxr}
This section presents WebXR. The information was mostly taken from~\citep{webapismdnwebxr}.

WebXR is an API for web content that uses augmented (AR) or virtual reality (VR). It was developed by Mozilla in 2014 under the name WebVR.
Later in 2018, with the addition of AR, it was merged into WebXR. Both AR and VR paradigms belong to the mixed reality (MR) or cross
reality (XR), hence the name. However, WebXR does not support rendering of objects. This is entirely up to the developer, that they can use the previously mentioned WebGL API\@.

WebXR and its features are not supported on all browsers and versions,\footnote{Information on which browsers support
WebXR and corresponding features can be found at \\ \url{https://developer.mozilla.org/en-US/docs/Web/API/WebXR_Device_API} and
\url{https://immersiveweb.dev}} so it is still in active development at the time of writing. This may present a limitation for some projects
because developers have to take into consideration the device type, the environment in which it will be used, etc.

As we already said, WebXR supports both augmented reality and virtual reality. Which one we want to use depends on the type of experience
we want to create for the user. Each paradigm requires a different approach and sometimes also a specific device for visualizing the project.

VR and AR paradigms are mapped to:
\begin{enumerate}
    \item Virtual reality -- the entire image we see is digitally created by the application. There are two VR modes available: \emph{inline} and
    \emph{immersive-vr}. The first one is rendered within the web browser and does not require special hardware device to display. The latter
    requires a headset to render the scene which can be seen by the user’s eyes.
    \item Augmented reality -- this mode renders the image in the real-world environment around us. Since this is always an immersive
    experience, we only have one mode -- \emph{immersive-ar}. Types of devices that can be used to explore AR are phones/tablets and transparent
    glasses.
\end{enumerate}

In the implementation part, in order to visualize the cube in AR mode, we used the latest Chrome version on an Android device and the
XRViewer application for an iPhone device.

For further reference and a deeper understanding of how to create an application in AR/VR, please refer to~\citep{baruah2021ar}.
