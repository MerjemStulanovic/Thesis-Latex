\section{Angular}\label{sec:angular}
This section will give a basic introduction to Angular. The information was mostly taken from~\citep{angular} and~\citep{murray2018ng}
which can be consulted for further reference.

Angular is an open-source framework for web applications built on TypeScript. It was developed by Google in 2010 and the first version was
known as AngularJS. At the time of writing, the latest version is Angular 15 released in November 2022.

Angular, as a development platform, includes:
\begin{enumerate}
    \item A framework based on components for creating web applications.
    \item A number of libraries that include many features such as routing, forms management, client-server communication, and others.
    \item A set of developer tools for developing, building, testing, and updating the code.
\end{enumerate}

Components are the building blocks of Angular architecture. We use them as custom-made HTML tags that have specific functionalities linked to
them.

\begin{figure}[hbt!]
    \begin{center}
        \begin{lstlisting}[language=JavaScript,label={lst:angular-code-1},belowskip=-1 \baselineskip]
import { Component } from '@angular/core';

@Component({
    selector: 'app-home',
    templateUrl: './home.component.html',
    styleUrls: ['./home.component.css']
})

export class HomeComponent {
    // ...
}
        \end{lstlisting}
    \end{center}
    \caption{Example of an Angular component}
    \label{fig:figure4.1}
\end{figure}

The \texttt{import} statement specifies which modules to use to write the code. In this case, we are importing the
\texttt{Component} module. Next, we need to declare the component. \texttt{@Component} is called a decorator - which can be thought of as
metadata added to the code. The \texttt{selector} inside acts as a new tag to use in the HTML file of the component, so the tag will be
\texttt{<app-home>}. The \texttt{templateUrl} selector loads the template from the \texttt{home.component.html} file. The \texttt{styleUrls}
loads the CSS file associated to style the component. By default, the style will only apply to the given component and this approach Angular
uses is called \emph{style-encapsulation}.

After the component is created, it is necessary to load it. In order to do that, it is mandatory to add the component tag to the related template
file.

\begin{figure}[hbt!]
    \begin{center}
        \begin{lstlisting}[language=HTML,label={lst:angular-code-2},belowskip=-1 \baselineskip]
<div>
    <app-home></app-home>
</div>
        \end{lstlisting}
    \end{center}
    \caption{Loading Angular component in HTML file}
    \label{fig:figure4.2}
\end{figure}

This was a basic introduction to Angular with an example of creating and loading a simple component. The next section will give us
insight into the Three.js library, also used in the frontend.